\documentclass{article}\usepackage[]{graphicx}\usepackage[]{xcolor}
% maxwidth is the original width if it is less than linewidth
% otherwise use linewidth (to make sure the graphics do not exceed the margin)
\makeatletter
\def\maxwidth{ %
  \ifdim\Gin@nat@width>\linewidth
    \linewidth
  \else
    \Gin@nat@width
  \fi
}
\makeatother

\definecolor{fgcolor}{rgb}{0.345, 0.345, 0.345}
\newcommand{\hlnum}[1]{\textcolor[rgb]{0.686,0.059,0.569}{#1}}%
\newcommand{\hlstr}[1]{\textcolor[rgb]{0.192,0.494,0.8}{#1}}%
\newcommand{\hlcom}[1]{\textcolor[rgb]{0.678,0.584,0.686}{\textit{#1}}}%
\newcommand{\hlopt}[1]{\textcolor[rgb]{0,0,0}{#1}}%
\newcommand{\hlstd}[1]{\textcolor[rgb]{0.345,0.345,0.345}{#1}}%
\newcommand{\hlkwa}[1]{\textcolor[rgb]{0.161,0.373,0.58}{\textbf{#1}}}%
\newcommand{\hlkwb}[1]{\textcolor[rgb]{0.69,0.353,0.396}{#1}}%
\newcommand{\hlkwc}[1]{\textcolor[rgb]{0.333,0.667,0.333}{#1}}%
\newcommand{\hlkwd}[1]{\textcolor[rgb]{0.737,0.353,0.396}{\textbf{#1}}}%
\let\hlipl\hlkwb

\usepackage{framed}
\makeatletter
\newenvironment{kframe}{%
 \def\at@end@of@kframe{}%
 \ifinner\ifhmode%
  \def\at@end@of@kframe{\end{minipage}}%
  \begin{minipage}{\columnwidth}%
 \fi\fi%
 \def\FrameCommand##1{\hskip\@totalleftmargin \hskip-\fboxsep
 \colorbox{shadecolor}{##1}\hskip-\fboxsep
     % There is no \\@totalrightmargin, so:
     \hskip-\linewidth \hskip-\@totalleftmargin \hskip\columnwidth}%
 \MakeFramed {\advance\hsize-\width
   \@totalleftmargin\z@ \linewidth\hsize
   \@setminipage}}%
 {\par\unskip\endMakeFramed%
 \at@end@of@kframe}
\makeatother

\definecolor{shadecolor}{rgb}{.97, .97, .97}
\definecolor{messagecolor}{rgb}{0, 0, 0}
\definecolor{warningcolor}{rgb}{1, 0, 1}
\definecolor{errorcolor}{rgb}{1, 0, 0}
\newenvironment{knitrout}{}{} % an empty environment to be redefined in TeX

\usepackage{alltt}
\usepackage{enumerate}
\usepackage{listings}

\newcommand{\rcode}[1]{\lstinline[language=R,basicstyle=\normalsize\ttfamily]!#1!}

\title{Fitting a zeta distribution to a {P} survey---number of groups data}
\IfFileExists{upquote.sty}{\usepackage{upquote}}{}
\begin{document}



In this example we will learn how to use \rcode{fitPS} to fit a zeta distribution to some data from a survey where the number of groups of glass found is recorded. The data in this example comes from @roux2001 who surveyed the footwear of 776 individuals in south-eastern Australia, and is summarised in the table below.
% latex table generated in R 4.3.2 by xtable 1.8-4 package
% Thu Dec 21 14:34:28 2023
\begin{table}[ht]
\centering
\begin{tabular}{rr}
  \hline
$n$ & $r_n$ \\ 
  \hline
0 & 754 \\ 
  1 & 9 \\ 
  2 & 8 \\ 
  3 & 4 \\ 
  4 & 1 \\ 
   \hline
\end{tabular}
\end{table}


This data set is built into the package and can be accessed from the \rcode{Psurveys} object. That is, we can type:
\begin{knitrout}
\definecolor{shadecolor}{rgb}{0.969, 0.969, 0.969}\color{fgcolor}\begin{kframe}
\begin{alltt}
\hlkwd{data}\hlstd{(}\hlstr{"Psurveys"}\hlstd{)}
\hlstd{roux} \hlkwb{=} \hlstd{Psurveys}\hlopt{$}\hlstd{roux}
\end{alltt}
\end{kframe}
\end{knitrout}
The data is stored as an object of class \rcode{psData}. This probably will not be of importance to most users. If you are interested in the details, then these can be found in the \textbf{Value} section of the help page for \rcode{readData}. There is an S3 \rcode{print} method for objects of time, meaning that if we print the object---either by typing its name at the command prompt, or by explicitly calling \rcode{print}---then we will get formatted printing of the information contained within the object. Specifically:
\begin{enumerate}[-]
  \item the values of $n$ and $r_n$ will be printed out in tabular format (where $n$ is the number of groups or the size of the groups, and $r_n$ is the number of times $n$ has been observed in the survey),
  \item the type of survey will be printed---either "Number of Groups" or "Group Size",
  \item and if the object has a reference or notes attached to it, then these will be
  printed as well
\end{enumerate}
For example
\begin{knitrout}
\definecolor{shadecolor}{rgb}{0.969, 0.969, 0.969}\color{fgcolor}\begin{kframe}
\begin{alltt}
\hlstd{roux}
\end{alltt}
\begin{verbatim}


Number of Groups

  n    rn
---  ----
  0   754
  1     9
  2     8
  3     4
  4     1
Roux C, Kirk R, Benson S, Van Haren T, Petterd C
(2001). "Glass particles in footwear of members of
the public in south-eastern Australia-a survey."
_Forensic Science International_, *116*(2), 149-156.
doi:10.1016/S0379-0738(00)00355-8
<https://doi.org/10.1016/S0379-0738%2800%2900355-8>.
\end{verbatim}
\end{kframe}
\end{knitrout}
It is very simple to fit a zeta distribution to this data set. We do this using the \rcode{fitDist} function.
\begin{knitrout}
\definecolor{shadecolor}{rgb}{0.969, 0.969, 0.969}\color{fgcolor}\begin{kframe}
\begin{alltt}
\hlstd{fit} \hlkwb{=} \hlkwd{fitDist}\hlstd{(roux)}
\end{alltt}
\end{kframe}
\end{knitrout}
The function returns an object of class \rcode{psFit}, the details of which can be found in the help page for \rcode{fitDist}. There are both S3 \rcode{print} and S3 \rcode{plot} methods for objects of this class. The \rcode{print} method method displays an estimate of the shape parameter $s$, an estimate of the standard deviation---the standard error---of the estimate of $s$ ($\widehat{\mathrm{sd}}(\hat{s})=\mathrm{se}(\hat{s})$). **NOTE**: it is important to understand that the value of the shape parameter that is displayed, and the value that is stored in the fitted object differ by 1. That is, $s$ is shown,m and $s^\prime = s - 1$ is stored. This is done because the package which assists in the fitting, \rcode{VGAM}, is parameterised in terms of $s^\prime$. This difference only has consequences if the fitted value is being used in conjuction with other \rcode{VGAM} functions.   The \rcode{print} method also displays the first 10 fitted probabilities from the model by default. The number of probabilities shown can be altered by changing the value of \rcode{nterms}. We will do this here to only display the first six probabilities (from $P_0$ to $P_5$).
\begin{knitrout}
\definecolor{shadecolor}{rgb}{0.969, 0.969, 0.969}\color{fgcolor}\begin{kframe}
\begin{alltt}
\hlkwd{print}\hlstd{(fit,} \hlkwc{nterms} \hlstd{=} \hlnum{4}\hlstd{)}
\end{alltt}
\begin{verbatim}
The estimated shape parameter is 4.9544 
The standard error of shape parameter is 0.2366 
------
NOTE: The shape parameter is reported so that it is consistent with Coulson et al.
However, the value returned is actually s' = shape - 1 to be consistent with the 
VGAM parameterisation, which is used for computation. This has flow on effects, for
example in confInt. This will be changed at some point.
------

The first  4 fitted values are:
         P0          P1          P2          P3 
0.963154667 0.031064468 0.004167082 0.001001917 
\end{verbatim}
\end{kframe}
\end{knitrout}
The package provides a \rcode{confint} method for the fitted value. The method returns both a Wald confidence interval and profile likelihood interval. The Wald interval takes the usual the form where the lower and upper bound are given by $\hat{s} \pm z^*(1-\alpha/2)\times se(\hat{s}))$. The profile likelihood interval finds the end-points of the interval that satisfies
\[
-2\left[\ell(\hat{s};\mathbf{x})-\ell(s;\mathbf{x})\right] \le \chi^2_1(\alpha)
\]
where $\ell(s;\mathbf{x})$ is value of the log-likelihood given shape parameter $s$. The two intervals are returned as elements of a \rcode{list} named \rcode{wald} and \rcode{prof} respectively.
\begin{knitrout}
\definecolor{shadecolor}{rgb}{0.969, 0.969, 0.969}\color{fgcolor}\begin{kframe}
\begin{alltt}
\hlstd{ci} \hlkwb{=} \hlkwd{confint}\hlstd{(fit)}
\hlstd{ci}\hlopt{$}\hlstd{wald}
\end{alltt}
\begin{verbatim}
    2.5%    97.5% 
3.490761 4.418099 
\end{verbatim}
\begin{alltt}
\hlstd{ci}\hlopt{$}\hlstd{prof}
\end{alltt}
\begin{verbatim}
    2.5%    97.5% 
3.520495 4.451277 
\end{verbatim}
\end{kframe}
\end{knitrout}
You will notice that neither of these intervals contain the value shown in the previous output. However, this simply because they are confidence intervals on $s^\prime$ and not $s$. This can be remedied by adding one to each interval:
\begin{knitrout}
\definecolor{shadecolor}{rgb}{0.969, 0.969, 0.969}\color{fgcolor}\begin{kframe}
\begin{alltt}
\hlstd{ci}\hlopt{$}\hlstd{wald} \hlopt{+} \hlnum{1}
\end{alltt}
\begin{verbatim}
    2.5%    97.5% 
4.490761 5.418099 
\end{verbatim}
\begin{alltt}
\hlstd{ci}\hlopt{$}\hlstd{prof} \hlopt{+} \hlnum{1}
\end{alltt}
\begin{verbatim}
    2.5%    97.5% 
4.520495 5.451277 
\end{verbatim}
\end{kframe}
\end{knitrout}
The reason for not *correcting* these intervals is that the method mostly exists to feed into other parts of the package, especially the \rcode{plot} method.
\end{document}
